\documentclass[12pt]{revtex4-1}

\usepackage[margin=1in]{geometry} % one inch margins
\usepackage{fancyhdr} % easier header and footer management
\pagestyle{fancyplain} % page formatting style
\fancyhf{} % clear headers and footers
\usepackage{hyperref} % for linking references
\setlength{\parindent}{0cm} % don't indent new paragraphs...
\parskip 6pt % ... place a space between paragraphs instead
\frenchspacing % add a single space after a period

\renewcommand{\headrulewidth}{0pt} % remove horizontal line in header

\usepackage{physics,bm}

\begin{document}

\begin{center}
  \large \bf Exploring the Hopfield model of neural networks
\end{center}

\begin{center}
  Tennesse Joyce, Joshua Levin, Michael A. Perlin
\end{center}

Neural networks provide a means of computing inspired by biological
brains. Modern neural networks consist of thousands to millions of
nodes corresponding to neurons, together with weighted edges which
make up the ``synapses'' of the network. A large number of degrees of
freedom makes neural networks amenable to statistical analysis.

There are several models of neural networks; we propose to study the
Hopfield model, popularized by John Hopfield by 1982. The Hopfield
model resembles a classical spin system: each node takes a value
$s_i\in\left\{1,-1\right\}$, and coupling constants $J_{ij}$
(corresponding to edges, or synapses) between spins determine an
``energy'' of any given network configuration, given by
\begin{align}
  E(\bm s) = -\frac12\sum_{i,j} J_{ij}s_is_j.
\end{align}
One can thus construct a partition function and apply methods of
statistical mechanics to study neural networks.

A recent paper by M\'ezard\cite{mezard2017mean} discusses a phase
diagram of the Hopfield model. In particular, M\'ezard studies a
so-called ``retrieval phase'' in which the network relaxes into one of
several predetermined states (which define the couplings $J_{ij}$),
thus exhibiting memory-like behavior.

We would like to study the different phases of the Hopfield model,
ideally numerically reproducing some results from M\'ezard's paper
(and perhaps others)\cite{amit1985storing,tanaka1998mean}. We will
start by writing Monte Carlo simulations of the Hopfield model to
explore its state space and compute various thermodynamic quantities
(e.g. internal energy, heat capacity, and configurational entropy) and
order parameters. If time allows, we also wish to investigate how
constraints on the coupling constants $J_{ij}$ affect the phase
diagram of the Hopfield model.

\bibliography{\jobname}

\end{document}
